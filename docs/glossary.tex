\chapter{Glossary}
\begin{description}
  
\item{\textbf{Branch}} \hfill \\ \textit{A thread of development within a project or team; branches are common in open source development (Robles and González-Barahona, 2012).}

\item{\textbf{Evolution}} \hfill \\ \textit{Defined in paragraph 1.3.}

\item{\textbf{Fork}} \hfill \\ \textit{Defined in paragraph 1.3.}

\item{\textbf{Merge}} \hfill \\ \textit{A rejoining of separate development strands that had branched or forked previously, either by integrating source code or by dismissing parts of either project (Robles and González-Barahona, 2012).}

\item{\textbf{Open source software}} \hfill \\ \textit{Software development paradigm that blurs the difference between users and developers (Hippel and Krogh, 2003). Open source software licenses grant users the right to fork a project (Robles and González-Barahona, 2012).}

\item{\textbf{Phylogenetic tree}} \hfill \\ \textit{A pictorial representation of the degree of relationship between entities sharing a common ancestry (Baum and Offner, 2008).}

\item{\textbf{Release}} \hfill \\ \textit{A stage in the software lifecycle corresponding to a new generation of the system (Lehmann, 1980).}

\end{description}
